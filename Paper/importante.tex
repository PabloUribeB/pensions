Solo el 15\% de las mujeres se pensionan, comparado a un 27\% en hombres. Esto podria explicar el porque no se encuentran efectos en mujeres. Tambien correlacionado con el hecho de que al tener menor edad de pension y cotizar menos semanas debido al childbearing, es mas dificil que accedan a la pension. Es un problema del sistema, pero explica la ausencia de efectos significativos. Aca para esas cifras y demas citar este paper:

https://books.google.com/books?hl=en&lr=&id=PbJOEQAAQBAJ&oi=fnd&pg=PA87&ots=e90pESQ6H9&sig=_EdQhV9WptJ-ZDmljaoHxGqYHS8#v=onepage&q&f=false

https://ideas.repec.org/p/col/000547/017568.html



Potencialmente mostrar ibc_cum pero tambien ibc normal. El cum muestra el patron negativo en ambos, mientras que el normal muestra el drop a los 60 solo que no tiene poder suficiente para ser significativo, pero podria hacer un pooled regression y estimo el coeficiente cuando tienen 60 para ver si me da significativo (esto siguiendo la especificacion de chat gpt). Y poner ese resultado en las notas de la figura. Ahi argumento que para tratar de aumentar el poder hago esa estimacion y el resultado da XX.